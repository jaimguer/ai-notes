
\documentclass{article}

\begin{document}

\section*{Introducing Mycin}
\textit{Mycin} was an expert (rule-based) system that would assist in the
treatment of blood infections.  It was believed that since antibiotic
prescritpion is difficult and requires some depth of knowledge, Mycin would be
able to quickly give suggestions and be able to explain its reasoning.

\section*{Mycin's Components}
Mycin had five primary components.
\begin{enumerate}
    \item Knowledge base: a list of facts about its domain that were provided
        by someone with the requisite knowledge.
    \item Patient database: a database consisting of patient history and
        current information about their case.
    \item Consultation module: a program where a doctor could ask Mycin
        questions, and it would respond with advice pertaining to the case.
    \item Explanation module: a program where Mycin would explain its
        reasoning about its choices.
    \item Knowledge aquisition module: Mycin would learn from its experiences
        and change its rules accordingly.
\end{enumerate}
Internally, Mycin used context trees to help organize its data and make decisions.

\section*{Evaluation}
Mycin's predictions were compared to experts in the field.  Its results were
comparable to expert recommendations and better than non-expert
recommendations.

\section*{Problems}
\begin{enumerate}
    \item Mycin's knowledge base was limited.
    \item Doctors were hesitant to trust it.
    \item Terrible user interface.
\end{enumerate}

\end{document}
